%!TEX root = ../main.tex

\subsection{Influence of receiver position on costs}
\newcommand{\gbcns}{\mathcal{C}}

One distinguishing feature of networks where player costs have improved after the suggestion is that the seeds advise to play a best response for a larger fraction of time. However, there is not a significantly higher fraction of receivers for which the seed is their current colour choice in the two successful networks. But by the end of the round, in \textit{ba} and \textit{ws}, a significantly larger fraction of receivers never deviated from the suggestion than in \textit{er} and \textit{sb}. In other words, their environment maintained the seed as a best response.

The preceding observation indicates that network effects are at play, such as the topology of the graphs and the position of the receivers.
The group betweenness centrality \citep{everett_centrality_1999} of a set of nodes is employed here to explain the qualitatively different results obtained in the four networks. Let \( \gbcns \) be a set of nodes in the network. Their group betweenness centrality (GBC) is measured by the sum over all pairs of sources and destinations \( s, \, t \) (not included in \( \gbcns \)) of the fraction of shortest paths connecting \( s \) to \( t \) running through at least one point in \( \gbcns \) among all shortest paths connecting \( s \) to \( t \).
% \[
% 	\text{GBC}(\gbcns) = \sum_{s,t \notin \gbcns} \frac{\sigma^*_{s,t}(\gbcns)}{\sigma_{s,t}}
% \]

\begin{figure}
  \centering
	\begin{subfigure}[t]{0.45\linewidth}
    \centering
	  \includegraphics[width=\linewidth]{Coloring/Figures/wslowgbc}
    \caption{}
  \end{subfigure}
	\hfill
  \begin{subfigure}[t]{0.45\linewidth}
    \centering
  	\includegraphics[width=\linewidth]{Coloring/Figures/wshighgbc}
    \caption{}
  \end{subfigure}
  \caption{Comparison of the group betweenness centrality of two sets of receivers with the sum of their individual betweenness centralities \( \Sigma_b \). The GBC more accurately measures the centrality of the set of receivers.
	\textbf{a.}~GBC = \( 63.84 \), \( \Sigma_b = 300.57 \).
	\textbf{b.}~GBC = \( 240.43 \), \( \Sigma_b = 286.8 \).}
  \label{fig:gbccomp}
\end{figure}

We compare in Figure \ref{fig:gbccomp} the GBC of a set of nodes with the sum of their individual betweenness centralities. The latter does not discriminate between a highly clustered set of receivers, for which the seeds may only have local effects, and a set that covers more appropriately the network. This is unlike the GBC which gives a starkly different measure for both sets, and is thus a relevant metric for the effect of receiver positions.

We test this hypothesis by sampling at random \( k \) receivers on the network. A dynamics is run until an equilibrium is reached, at which point a colouring is selected in the same manner as in the game. The \( k \) receivers immediately update to the seed, and the dynamics is run again until a new equilibrium \( S_f \) is reached. We repeat this procedure for \( m \) times and average the number of matches obtained in \( S_f \). The set of receivers is sampled \( n \) times, for \( k = \{3, 6\} \) and we compare the series of receivers' GBC with the corresponding average number of matches, using Spearman's rank correlation coefficient. The full table of results is presented in Table \ref{tab:col/gbcsim}.

\begin{table}
  \centering
  \caption{Spearman rank correlation coefficient between group betweenness centrality and average number of matches at final equilibrium, for each network and communication level. Significance levels are given next to the values: (*) \( p < 0.05 \), (**) \( p < 0.01 \), (***) \( p< 0.001 \).
  Best response dynamics are run by sampling one node at a time and choosing at random a best response, until all nodes are at equilibrium.
  As suggestions are sent, receivers follow the seed unconditionally. Multiplicative weight updates (MWU) keeps track of the players' mixed strategies.
  Finally, we test the correlation between the GBC of the set of receivers and the average player cost for simulations based on the behavioural model of players.
  Each dynamics is run for \( n \) times on \( m \) random receiver sets.
  }
  \label{tab:col/gbcsim}
  \begin{tabular}{l|l|llll}
  & & \multicolumn{4}{c}{Networks} \\
  \midrule
  Dynamics & Receivers & ba & ws & er & sb \\
  \midrule
  \multirow{2}{*}{\shortstack[l]{BR \\ \( (n = 200, m = 100) \)}} & 10\% &
      -0.355 (***) & -0.215 (**) & -0.144 (*) & -0.107 \\
    & 20\% & -0.449 (***) & -0.137 & -0.279 (***) & -0.391 (***) \\
  \midrule
  \multirow{2}{*}{\shortstack[l]{MWU \\ \( (n = 100, m = 100) \)}} & 10\% &
      -0.220 (*) & -0.282 (**) & 0.007 & -0.166 \\
    & 20\% & -0.119 & -0.112 & -0.099 & 0.113 \\
  \midrule
  & & & & & \\
  \midrule
  \multirow{2}{*}{\shortstack[l]{Behavioural model \\ \( (n = 200, m = 100) \)}} & 10\% &
      -0.549 (***) & -0.176 (*) & -0.363 (***) & -0.419 (***) \\
    & 20\% & -0.592 (***) & -0.236 (***) & -0.352 (***) & -0.553 (***) \\
  \bottomrule
  \end{tabular}
\end{table}

With best response (BR) dynamics, for almost all networks and all values of \( k \), the coefficient is significantly smaller than 0, indicating that a higher GBC translates to a lower average number of matches. The dynamics is obtained by sampling uniformly at random one node at a time and allowing this node to play her current best response strategy, or one of them in case of ties.

The results are more contrasted with multiplicative weight update (MWU) dynamics \citep{littlestone1994weighted}, perhaps reflecting the noisier approach of the procedure. The dynamics is carried over mixed strategies of the agents, or probability distribution over their colour choices. At each step, every player's mixed strategy is updated by decreasing the probability of a colour that yields a higher cost against the expected choice of one's neighbours.

Finally, a behavioural model obtained from the experimental data (detailed in \ref{sec:col/simdef}) yields strong indication that the GBC of receivers is inversely correlated with the average of all player costs, computed with the same experimental scoring function. The model is obtained from two logistic regressions encoding respectively the choice to deviate from one's current action and the subsequent colour choice.

\begin{figure}
  \centering
  \includegraphics[width=0.7\linewidth]{Coloring/Figures/netcorr-box.pdf}
  \caption{Spearman correlation coefficient for networks of 20 nodes and three dynamics (BR, MWU, behavioural model). For each network, 25 sets of 2 and 4 receivers are selected randomly. For each set of receivers, each dynamics is simulated 50 times and its resulting statistic is computed, the average number of matches for BR and MWU and the resulting average player cost for the behavioural model. The correlation is obtained for each network between the GBC of the receiver set and the resulting statistic. The boxplot represents the distribution of these coefficients for each network family and number of receivers. For most networks, the correlation coefficient is significantly below 0, indicating that more central receivers (as measured by the GBC) tend to improve the play.}
	\label{fig:netgbc}
\end{figure}
